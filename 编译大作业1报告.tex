\documentclass[a4paper,11pt]{ctexart}
%中文论文
\title{编译技术Project1报告}
\author{刘皓华 ~黄凯 ~叶文沁 ~王胤博}
\date{ }
\usepackage{amsmath}
\usepackage{CJK}
\usepackage{amsthm}
\usepackage{amssymb}%\geqslant
\usepackage{graphicx}
\usepackage{booktabs}%表格中画横线
\usepackage{multirow}%表格行列合并
\usepackage{caption}%表格标题
\usepackage{listings}
\usepackage{cases}
\usepackage{enumerate}
\usepackage{ctex}
\begin{document}
\maketitle
\subsection*{1. 分工}
叶文沁:solution.cc,buildTable.h。

黄凯:词法分析和语法分析。

刘皓华:代码生成。

王胤博:整合形成报告。
\subsection*{2. 思路与实现}
\subsubsection*{整体思路}

从json文件里提取kernel,为之建立DAG,然后生成符号表,建立IR树,生成代码。

\subsubsection*{词法分析和语法分析}

在进行词法分析之前,需要提取json文件中的kernel字段作为词法分析程序的输入。由于本次project的json文件格式固定且较为简单,没有使用第三方工具jsoncpp,而是自己编写了一个读取json文件的函数。该函数会读取json文件的各个字段,将其存储在预先定义好的结构体中,并去除kernel字段里多余的空格。读取完成后,将kernel字段存储在一个中间文件中,以该中间文件作为词法分析程序的输入。我们的词法分析程序借助词法分析工具Lex实现,语法分析程序借助语法分析工具Yacc实现。词法分析程序向语法分析程序提供token,语法分析程序在每次需要时调用词法分析程序来获取token。为了实现的方便,我们对RHS的文法进行了调整,将其改写成了$RHS\rightarrow RHS+Term|RHS-Term$的形式。借助词法分析程序和语法分析程序,为输入的kernel建立了一个DAG,供代码生成使用。

\subsubsection*{solution.cc}

在分析json文件和生成DAG后,首先遍历一遍DAG,生成符号表(代码在buildTable.h,其中的indexDom记录每个index的范围,varTable记录每个数组的大小信息。通过遍历Tref或Sref节点的左子节点记录数组大小,将信息放入varTable。遍历Tref节点的右子节点,计算下标范围。将每个表达式范围作为继承属性传递给子表达式,并计算子表达式的范围,最后在进入index节点后更新indexDom中该index的范围),然后建立IR树(基本思路:每个等式对应有一个和LHS相同大小的临时变量,用于临时记录计算结果。每一个等式会产生一个临时变量声明语句、一个计算临时变量的循环语句、一个将临时变量赋给LHS的循环语句。每个等式的RHS分成若干Term处理,每个Term对应一条语句。如果这个Term有不在LHS的index,那么该语句就是一条用于求和的循环语句,否则是一条给临时变量赋值的语句)。

\subsubsection*{代码生成}

1. 新的IR节点类:向IR中添加了一种语句类型,用于变量的声明,命名为Def。Def类的实例里只包含一个Expr类实例,在创造一个Def实例时只需要一个包含了该声明语句的变量的所有信息的Var类的实例。

附上Def的定义:
\begin{lstlisting}[breaklines,language=c++]
class Def: public StmtNode, public std::enable_shared_from_this<Def> {
 public:
    Expr var;
    //std::vector<Expr> index_list;
    //std::vector<Stmt> body_list;

    Def(Expr _var) :
        StmtNode(IRNodeType::Def), var(_var) {}

    Stmt mutate_stmt(IRMutator *mutator) const;
    void visit_node(IRVisitor *visitor) const;

    static Stmt make(Expr _var) {
        return std::make_shared<const Def>(_var);
    }

    static const IRNodeType node_type_ = IRNodeType::Def;
};
\end{lstlisting}

2. IRPrinter:IRPrinter主要是深度优先遍历IR树,并在必要的节点进行打印。其中较为重要的细节包括:
\begin{enumerate}[1)]
\item 对于每一个运算,在打印时都用括号封装起来,保证c程序运行时不出错。
\item 对于结构Var,在打印参数、打印声明语句与打印普通变量时,使用三种不同的处理,用在IRPrinter.h里定义的两个变量控制。
\item 在Var的处理中,需要特别注意区分标量,用shape进行判断时,考虑是不是shape的大小为1且第一个的值也为1;用args判断时,只要判断args的大小即可。
\item 在kernel里要注意一下没有输入参数的情况。
\end{enumerate}
\end{document}
